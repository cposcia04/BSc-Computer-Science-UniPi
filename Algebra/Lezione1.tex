\section{Lezione 1}
\begin{lezionebox}
\begin{itemize}
  \item Principio di induzione
  \item Principio di induzione in forma completa
  \item Principio di buon ordinamento
  \item Equivalenza dei tre principi
\end{itemize}
\end{lezionebox}

Iniziamo enunciando tre principi fondamentali dell'algebra: il \emph{principio di induzione}, il \emph{principio di induzione in forma completa} e il \emph{principio del buon ordinamento}. In seguito dimostreremo che questi tre principi sono logicamente equivalenti.



\begin{definitionbox}
\textbf{Principio di induzione.} Sia $S \subseteq \mathbb{N}$ tale che:
\begin{enumerate}[label=(\roman*)]
    \item $0 \in S$;
    \item $n \in S \Rightarrow n+1 \in S$ per ogni $n \in \mathbb{N}$.
\end{enumerate}
Allora $S = \mathbb{N}$.
\end{definitionbox}

La definizione precedente può anche essere espressa in forma simbolica:
\[
\left(0 \in S \subseteq \mathbb{N} \land \forall n \in \mathbb{N} \, (n \in S \Rightarrow n+1 \in S)\right) \Rightarrow S = \mathbb{N}
\]

\begin{examplebox}
\textbf{Esempio } Dimostrare per induzione la formula:
\[
\forall n \in \mathbb{N}, \quad \sum_{i=0}^n i = \frac{n(n+1)}{2}
\]

Definiamo $P(n) \leftrightarrow \sum_{i=0}^n i = \frac{n(n+1)}{2}$ e consideriamo $S = \{n \in \mathbb{N} : P(n)\ \text{è vera} \}$.

\begin{itemize}
    \item \textbf{Caso base:} 
    \[
    \sum_{i=0}^0 i = 0 = \frac{0\cdot(0+1)}{2} \Rightarrow 0 \in S
    \]

    \item \textbf{Passo induttivo:} Supponiamo $n \in S$, ovvero $P(n)$ vera:
    \[
    \sum_{i=0}^{n+1} i = \sum_{i=0}^n i + (n+1) = \frac{n(n+1)}{2} + (n+1)
    = \frac{(n+1)(n+2)}{2}
    \]
    quindi $n+1 \in S$.
\end{itemize}

Per il principio di induzione, $S = \mathbb{N}$ e dunque la formula è vera per ogni $n \in \mathbb{N}$.
\end{examplebox}



\begin{definitionbox}
\textbf{Principio di induzione in forma completa.} Sia $S \subseteq \mathbb{N}$ tale che:
\begin{enumerate}[label=(\roman*)]
    \item $0 \in S$;
    \item Se $0,1,\dots,n \in S$, allora $n+1 \in S$ per ogni $n \in \mathbb{N}$.
\end{enumerate}
Allora $S = \mathbb{N}$.
\end{definitionbox}

Notiamo che le ipotesi del principio in forma completa sono  più deboli, ma il risultato resta invariato. IPOTESI PI IMPLICANO IPOTESI PIFC MA NON TUTTA L'IMPLICAZIONE AGGIUNGERE

\begin{definitionbox}
\textbf{Principio di buon ordinamento.} Ogni sottoinsieme non vuoto $S \subseteq \mathbb{N}$ ammette un elemento minimo.
\end{definitionbox}


\begin{theorembox}
\textbf{Equivalenza dei tre principi} \quad
Principio di induzione $\iff$ Principio di induzione in forma completa $\iff$ Principio di buon ordinamento.
\end{theorembox}

\textit{Dimostrazione.} \quad \begin{enumerate}
    \item [(i)]\textit{Principio di induzione} $\Rightarrow$ \textit{Principio di induzione forte} \\ 
    
    Dobbiamo dimostrare che, supponendo che  $\left(0 \in S \subseteq \mathbb{N} \land \forall n \in \mathbb{N} \, (0,...,n \in S \Rightarrow n+1 \in S)\right) $ e che vale il principio di induzione, allora $S=\mathbb{N}$. \\
    Definiamo un insieme $S'=\{s\in S \mid 0,...,s \in S\}$ che è un sottoinsieme di $S$, cioè $S'\subseteq S$. Facciamo vedere che $S'=\mathbb{N}$ tramite il principio di induzione, cioè che $S'$ ha le due proprietà che consentono di applicare il principio di induzione. Notiamo che $0 \in S'$
    
\end{enumerate}
